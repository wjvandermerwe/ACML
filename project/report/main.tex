

\documentclass[
12pt, % The default document font size, options: 10pt, 11pt, 12pt
oneside, % Two side (alternating margins) for binding by default, uncomment to switch to one side
english, % ngerman for German
onehalfspacing, % One-and-a-half line spacing, alternatives: singlespacing or doublespacing
%draft, % Uncomment to enable draft mode (no pictures, no links, overfull hboxes indicated)
nolistspacing, % If the document is onehalfspacing or doublespacing, uncomment this to set spacing in lists to single
liststotoc, % Uncomment to add the list of figures/tables/etc to the table of contents
%toctotoc, % Uncomment to add the main table of contents to the table of contents
%parskip, % Uncomment to add space between paragraphs
%nohyperref, % Uncomment to not load the hyperref package
headsepline, % Uncomment to get a line under the header
%chapterinoneline, % Uncomment to place the chapter title next to the number on one line
%consistentlayout, % Uncomment to change the layout of the declaration, abstract and acknowledgements pages to match the default layout
]{ProposalAndThesis} % The class file specifying the document structure

\usepackage[utf8]{inputenc} % Required for inputting international characters
\usepackage[T1]{fontenc} % Output font encoding for international characters

\usepackage{mathpazo} % Use the Palatino font by default

\usepackage[backend=bibtex,style=numeric,natbib=true]{biblatex} % Use the bibtex backend with the numeric citation style

\addbibresource{example.bib} % The filename of the bibliography




\usepackage{listings}
\usepackage{xcolor}
\usepackage{hyperref}
\usepackage[autostyle=true]{csquotes} % Required to generate language-dependent quotes in the bibliography

\usepackage[cleanlook, english]{isodate} % Required for UK date formatting

\usepackage{fancybox} % Required for boxed text sections
\usepackage{xcolor} % Required for coloured text

\usepackage{pgfgantt}

\usepackage[utf8]{inputenc}
\usepackage{graphicx}
\usepackage{amsmath}
\usepackage{hyperref}


\definecolor{ShadowColor}{RGB}{0,103,165} % Required for coloured shadow in boxed text sections
\makeatletter
\newcommand\Cshadowbox{\VerbBox\@Cshadowbox}
\def\@Cshadowbox#1{%
	\setbox\@fancybox\hbox{\fbox{#1}}%
	\leavevmode\vbox{%
		\offinterlineskip
		\dimen@=\shadowsize
		\advance\dimen@ .5\fboxrule
		\hbox{\copy\@fancybox\kern.5\fboxrule\lower\shadowsize\hbox{%
				\color{ShadowColor}\vrule \@height\ht\@fancybox \@depth\dp\@fancybox \@width\dimen@}}%
		\vskip\dimexpr-\dimen@+0.5\fboxrule\relax
		\moveright\shadowsize\vbox{%
			\color{ShadowColor}\hrule \@width\wd\@fancybox \@height\dimen@}}}
\makeatother

%----------------------------------------------------------------------------------------
%	MARGIN SETTINGS
%----------------------------------------------------------------------------------------

\geometry{
	paper=a4paper, % Change to letterpaper for US letter
	inner=2.8cm, % Inner margin
	outer=2.8cm, % Outer margin
	bindingoffset=0.0cm, % Binding offset
	top=2.8cm, % Top margin
	bottom=2.8cm, % Bottom margin
	%showframe, % Uncomment to show how the type block is set on the page
}

%----------------------------------------------------------------------------------------
%	THESIS INFORMATION
%----------------------------------------------------------------------------------------

\thesistitle{Analysis and Implementation of a Transformer Model for Translation} % Your thesis title, this is used in the title and abstract, print it elsewhere with \ttitle
% \supervisor{Name of Supervisor Here} % Your supervisor's name, this is used in the title page, print it elsewhere with \supname
% \cosupervisor{Name of Co-supervisor Here (or Delete)} % Your co-supervisor's name, this is used in the title page, print it elsewhere with \cosupname
%\examiner{} % Your examiner's name, this is not currently used anywhere in the template, print it elsewhere with \examname
\degree{Degree Name Here} % Your degree name, this is used in the title page and abstract, print it elsewhere with \degreename
\author{Your Name Here} % Your name, this is used in the title page and abstract, print it elsewhere with \authorname
\date{27 May 2024}

%\addresses{} % Your address, this is not currently used anywhere in the template, print it elsewhere with \addressname
%\subject{} % Your subject area, this is not currently used anywhere in the template, print it elsewhere with \subjectname
%\keywords{} % Keywords for your thesis, this is not currently used anywhere in the template, print it elsewhere with \keywordnames
\university{University of the Witwatersrand, Johannesburg} % Your university's name, this is used in the title page and abstract, print it elsewhere with \univname
\department{Name of School or Department Here} % Your department's name, this is used in the title page and abstract, print it elsewhere with \deptname
%\group{\href{http://research group.com}{Research Group Name}} % Your research group's name and URL, this is not currently used anywhere in the template, print it elsewhere with \groupname
%\faculty{\href{http://faculty.university.com}{Faculty Name}} % Your faculty's name and URL, this is not currently used anywhere in the template, print it elsewhere with \facname

\def\keywordnames{Appendices; Chapters; Figures; example.bib; main.pdf; main.tex; main.bbl; main.aux; main.blg; main.lof; main.log; main.lot; main.out; MastersDoctoralThesis.cls}

\newcommand{\keyword}[1]{\textbf{#1}}
\newcommand{\tabhead}[1]{\textbf{#1}}
\newcommand{\code}[1]{\texttt{#1}}
\newcommand{\file}[1]{\texttt{\bfseries#1}}
\newcommand{\option}[1]{\texttt{\itshape#1}}



\AtBeginDocument{
\hypersetup{pdftitle=\ttitle} % Set the PDF's title to your title
\hypersetup{pdfauthor=\authorname} % Set the PDF's author to your name
\hypersetup{pdfkeywords=\keywordnames} % Set the PDF's keywords to your keywords
}

\begin{document}

\frontmatter % Use roman page numbering style (i, ii, iii, iv...) for the pre-content pages

\pagestyle{plain} % Default to the plain heading style until the thesis style is called for the body content

%----------------------------------------------------------------------------------------
%	TITLE PAGE
%----------------------------------------------------------------------------------------
\lstdefinestyle{mystyle}{
    backgroundcolor=\color{gray!10},   
    commentstyle=\color{green},
    keywordstyle=\color{magenta},
    numberstyle=\tiny\color{gray},
    stringstyle=\color{red},
    basicstyle=\ttfamily\footnotesize,
    breakatwhitespace=false,         
    breaklines=true,                 
    captionpos=b,                    
    keepspaces=true,                 
    numbers=left,                    
    numbersep=5pt,                  
    showspaces=false,                
    showstringspaces=false,
    showtabs=false,                  
    tabsize=2
}
\lstset{style=mystyle}

\begin{titlepage}
\begin{center}

{\huge \bfseries \ttitle}\par\vspace{0.4cm} % Thesis title
\HRule\par\vspace{1.5cm}
\authorname\par\vspace{1cm}
\emph{Supervisor(s):}\par
% {\supname}\par % Name of supervisor
% {\cosupname} % Name of co-supervisor
\par\vspace{0.5cm}

\includegraphics[width=80mm]{Figures/logoWitsstackedcolourtransparent.png} % University crest
\vfill

A research proposal submitted in partial fulfillment of the requirements for the degree of \degreename\par\vspace{0.3cm}
in the\par\vspace{0.4cm}
\deptname\par\vspace{0.1cm} % Name of department
\univname\par\vspace{0.4cm} % Name of university
\cleanlookdateon
\today % Date

\end{center}

\end{titlepage}

\par\vspace{2cm}
\begin{flushright}
\includegraphics[width=30mm]{Figures/sign.png}\par 
\authorname\par\vspace{0.1cm}
\today
\end{flushright}

\vfill
\pagebreak

\tableofcontents % Prints the main table of contents

\listoffigures % Prints the list of figures

\mainmatter % Begin numeric (1,2,3...) page numbering


\section{Introduction}
This section provides a detailed explanation of the Python code used for training a Transformer model for language translation. The code utilizes PyTorch and other auxiliary libraries for handling datasets, preprocessing, and training procedures.

\section{Code Overview}
The script is structured to perform dataset loading, preprocessing, model training, validation, and saving checkpoints. Key functionalities are encapsulated in functions and executed conditionally.

\section{Detailed Code Explanation}
\subsection{Configuration and Setup}
\begin{lstlisting}[language=Python]
from torch.utils.data import DataLoader
from dataset import BilingualDataset
from model import build_transformer
from config import get_config
import torch
from pathlib import Path
from datasets import load_dataset
\end{lstlisting}
Libraries and modules necessary for dataset handling, model operations, and configurations are imported. The script checks for GPU availability and sets the device accordingly for training.

\subsection{Data Preparation}
\subsubsection{Load and Preprocess Function}
\begin{lstlisting}[language=Python]
def load_and_preprocess(config):
    ds_raw = load_dataset(f"{config['datasource']}", f"{config['lang_src']}-{config['lang_tgt']}", split='train')
    tokenizer_src = get_or_build_tokenizer(config, ds_raw, config['lang_src'])
    ...
    return train_dataloader, val_dataloader, tokenizer_src, tokenizer_tgt
\end{lstlisting}
This function handles the loading of the dataset and its preprocessing. It initializes tokenizers for both the source and target languages, splits the dataset, and prepares data loaders.

\subsection{Model Training}
\subsubsection{Training Function}
\begin{lstlisting}[language=Python]
\end{lstlisting}

\end{document}

\appendix

\printbibliography[heading=bibintoc]

%----------------------------------------------------------------------------------------

\end{document}  
